\documentclass{sigchi}

% Use this command to override the default ACM copyright statement (e.g. for preprints). 
% Consult the conference website for the camera-ready copyright statement.


%% EXAMPLE BEGIN -- HOW TO OVERRIDE THE DEFAULT COPYRIGHT STRIP -- (July 22, 2013 - Paul Baumann)
% \toappear{Permission to make digital or hard copies of all or part of this work for personal or classroom use is 	granted without fee provided that copies are not made or distributed for profit or commercial advantage and that copies bear this notice and the full citation on the first page. Copyrights for components of this work owned by others than ACM must be honored. Abstracting with credit is permitted. To copy otherwise, or republish, to post on servers or to redistribute to lists, requires prior specific permission and/or a fee. Request permissions from permissions@acm.org. \\
% {\emph{CHI'14}}, April 26--May 1, 2014, Toronto, Canada. \\
% Copyright \copyright~2014 ACM ISBN/14/04...\$15.00. \\
% DOI string from ACM form confirmation}
%% EXAMPLE END -- HOW TO OVERRIDE THE DEFAULT COPYRIGHT STRIP -- (July 22, 2013 - Paul Baumann)


% Arabic page numbers for submission. 
% Remove this line to eliminate page numbers for the camera ready copy
% \pagenumbering{arabic}


% Load basic packages
\usepackage{balance}  % to better equalize the last page
\usepackage{graphics} % for EPS, load graphicx instead
\usepackage{times}    % comment if you want LaTeX's default font
\usepackage{url}      % llt: nicely formatted URLs

% llt: Define a global style for URLs, rather that the default one
\makeatletter
\def\url@leostyle{%
  \@ifundefined{selectfont}{\def\UrlFont{\sf}}{\def\UrlFont{\small\bf\ttfamily}}}
\makeatother
\urlstyle{leo}


% To make various LaTeX processors do the right thing with page size.
\def\pprw{8.5in}
\def\pprh{11in}
\special{papersize=\pprw,\pprh}
\setlength{\paperwidth}{\pprw}
\setlength{\paperheight}{\pprh}
\setlength{\pdfpagewidth}{\pprw}
\setlength{\pdfpageheight}{\pprh}

% Make sure hyperref comes last of your loaded packages, 
% to give it a fighting chance of not being over-written, 
% since its job is to redefine many LaTeX commands.
\usepackage[pdftex]{hyperref}
\hypersetup{
pdftitle={SIGCHI Conference Proceedings Format},
pdfauthor={LaTeX},
pdfkeywords={SIGCHI, proceedings, archival format},
bookmarksnumbered,
pdfstartview={FitH},
colorlinks,
citecolor=black,
filecolor=black,
linkcolor=black,
urlcolor=black,
breaklinks=true,
}

% create a shortcut to typeset table headings
\newcommand\tabhead[1]{\small\textbf{#1}}


% End of preamble. Here it comes the document.
\begin{document}

\title{Kinesiological Control of Teleoperated \\Robotic Manipulators}

\numberofauthors{3}
\author{
  \alignauthor Christopher Bodden\\
    \affaddr{University of Wisconsin-Madison}\\
    \affaddr{1210 W. Dayton St.\\ Madison, WI 53706}\\
    \email{cbodden@cs.wisc.edu}
  \alignauthor Danny Rakita\\
    \affaddr{University of Wisconsin-Madison}\\
    \affaddr{1210 W. Dayton St.\\ Madison, WI 53706}\\
    \email{rakita@cs.wisc.edu}
  \alignauthor Alper Sarikaya\\
    \affaddr{University of Wisconsin-Madison}\\
    \affaddr{1210 W. Dayton St.\\ Madison, WI 53706}\\
    \email{sarikaya@cs.wisc.edu}
}

\maketitle

\begin{abstract}
While robots have the potential for increasing dexterity for completing manufacturing tasks, the lack of naturalistic robot manipulation make it difficult for non-experts to control the robot for a desired task. We explore the question of naturalistic control of a robot arm for tasks that have one-to-one mapping to natural human movement, and those that do not against typical joystick control. Through our exploration, we have developed a system that uses computer vision and a data glove to determine an individual's hand position and orientation in three-dimensional space. We map this position and orientation input to the robot's end effector. We show that using a combination of kinesiological control plus specialty functions can achieve similar performance and user perceptions to a joystick. Additionally, kinesiological control without direct mapping to the full range of the robot's actuated motion impairs performance for tasks that could benefit from it. Interestingly, task type had a strong effect on user perception of the control method. We note that further research in this area is needed to fully explore the potential of kinesiological control of co-working robots.
\end{abstract}

\keywords{
  Robot manipulation, naturalistic control, user study
}

\category{H.5.m.}{Information Interfaces and Presentation (e.g. HCI)}{Miscellaneous}

\section{Introduction}

Manipulation of external robot interfaces can be a challenging tasks, and moreso to enable novice controllers to control actuated motion with dexterity.  The multitude of methods and techniques for manipulating many robot actuators to accomplish seemingly simple tasks generally require extensive training and practice.  Many typical control methods for actuated robots require the operator to make a cognitive map from control space to the three-dimensional space in which the robot operates.  In our work, we explore kinesological control of a robot arm for novice users, looking at precision tasks that have a one-to-one mapping with human motion and those that do not.  We constract this kinesological control with conventional joystick manipulation, which provides more direct input to the robot.  

In the implementation of our work, we have developed a system that uses computer vision to track an individual's arm in three-dimensional space and map that input to the end effector of a Kinova \textbf{[name?]} robot arm.  Our kinesological implementation allows for real-time manipulation and control of the robot arm in real-time, and in conjunction with gyroscope readings from a smart glove worn by the individual, allows a user to move, rotate, and actuate fingers of the hand without direct manipulation of a joystick. In our experimentation, we also allow users to use the Kinova-included joystick to accomplish similar tasks. Through our study, we show how tasks that do not map one-to-one with natural movement can have limitations that need non-kinesological (e.g. direct) control to overcome.  However, we also show that depending on the task type, the control type had a strong effect on user perception.


\section{Background}

Advances in the areas of computer vision and robot control have enabled us to able to rapidly prototype and se these techniques to allow for novice control of a robot arm.  In particular, we utilized the Kinect for Windows SDK~\cite{kinectSDK} to quickly capture an individuals arm with a Microsoft Kinect.  By piping hand locations over the network to a machine running the ROS (Robot Operating System) system~\cite{ROS} interfaced to the Kinova robot arm, we are able to directly control the robot using kinesological control.

Several prior studies have looked at how remote control of robots can assist in manufacturing and human-robot collaboration tasks.  \textbf{[fill in rest of paragraph with cites!]}

In constructing our qualitative scales for post-hoc questionnaires, we used two seminal studies to report user preferences.  Davis~\cite{Davis1989} presents a methodological approach to measuring participants' perceived usefulness, ease-of-use, and acceptance of information interfaces. Although we adapt this to robot-human interaction, this adaptation of this scale to other interfaces has some precedence---Lin, \emph{et al.}~\cite{Lin2008} uses such an adaptation to measure interaction and perception of experience on the Internet.  We use the scale to measure user perceived usefulness, ease-of-use, and acceptance of the control method with regard to each task performed.


\section{Experimental Study}

\textbf{[set up study, what components we used, the task types, the experimental design, stats about the participant]}

\subsection{Procedure}

\textbf{[what measures we did, how we split up the participants, the procedure that each participant saw]}

\subsection{Results and Discussion}
\subsubsection{Task type}

\subsubsection{User perception}

\section{Future Work \& Conclusion}


% \begin{figure}[!h]
% \centering
% \includegraphics[width=0.9\columnwidth]{Figure1}
% \caption{With Caption Below, be sure to have a good resolution image
  % (see item D within the preparation instructions).}
% \label{fig:figure1}
% \end{figure}

% \begin{table}
  % \centering
  % \begin{tabular}{|c|c|c|}
    % \hline
    % \tabhead{Objects} &
    % \multicolumn{1}{|p{0.3\columnwidth}|}{\centering\tabhead{Caption --- pre-2002}} &
    % \multicolumn{1}{|p{0.4\columnwidth}|}{\centering\tabhead{Caption --- 2003 and afterwards}} \\
    % \hline
    % Tables & Above & Below \\
    % \hline
    % Figures & Below & Below \\
    % \hline
  % \end{tabular}
  % \caption{Table captions should be placed below the table.}
  % \label{tab:table1}
% \end{table}


\section{Acknowledgments}

We thank Professor Bilge Multu and Michael Gleicher for discussions in designing the user study, and also thank our participants for their in-depth comments and enthusiasm for the system.  The authors are supported by a number of different grants (ha!).

% Balancing columns in a ref list is a bit of a pain because you
% either use a hack like flushend or balance, or manually insert
% a column break.  http://www.tex.ac.uk/cgi-bin/texfaq2html?label=balance
% multicols doesn't work because we're already in two-column mode,
% and flushend isn't awesome, so I choose balance.  See this
% for more info: http://cs.brown.edu/system/software/latex/doc/balance.pdf
%
% Note that in a perfect world balance wants to be in the first
% column of the last page.
%
% If balance doesn't work for you, you can remove that and
% hard-code a column break into the bbl file right before you
% submit:
%
% http://stackoverflow.com/questions/2149854/how-to-manually-equalize-columns-
% in-an-ieee-paper-if-using-bibtex
%
% Or, just remove \balance and give up on balancing the last page.
%
\balance

\section{References format}
References must be the same font size as other body text.
% REFERENCES FORMAT
% References must be the same font size as other body text.

\bibliographystyle{acm-sigchi}
\bibliography{robot-control}
\end{document}
